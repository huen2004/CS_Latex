\ChapterImageStar[cap:disenio]{Diseño de la solución}{./images/fondo.png}\label{cap:disenio}
\mbox{}\\
\section{Modelado del sistema en Archimate}
ArchiMate es un lenguaje de modelado estandarizado por~\textit{The Open Group} que permite representar de manera estructurada y clara las diferentes capas de una arquitectura empresarial: negocio, aplicación y tecnología. Su propósito es brindar una visión integrada que facilite la comunicación entre los distintos actores de un proyecto y que muestre cómo los procesos de negocio, los sistemas de información y la infraestructura tecnológica se relacionan entre sí.
En particular, ArchiMate se organiza en vistas que permiten enfocarse en aspectos específicos: la vista de negocio describe los procesos y actores implicados, la vista de aplicación se centra en los sistemas de software que apoyan esos procesos, y la vista de tecnología aborda la infraestructura que soporta todo el ecosistema. Gracias a este enfoque por capas, los diagramas ayudan a identificar dependencias, puntos de optimización y la coherencia general de la solución arquitectónica.

\subsection{Vista de negocio}
\input{tablas-images/cp6/vista-negocio.tex}

\subsection{Vista de aplicación}
\input{tablas-images/cp6/vista-application.tex}

\subsection{Vista de tecnología}
\input{tablas-images/cp6/vista-tecnologia.tex}x


\subsection{Vista general}
% Vista general - Placeholder
% Este archivo contiene la vista general del modelo ArchiMate

\begin{figure}[H]
	\centering
	\includegraphics[width=\textwidth]{images/placeholder.png}
	\caption{Vista general del sistema}
	\label{fig:vista-general}
\end{figure}

% Descripción de la vista general
La vista general integra las tres capas del modelo ArchiMate (negocio, aplicación y tecnología) para proporcionar una perspectiva completa de la arquitectura de la solución. Esta vista permite visualizar las relaciones y dependencias entre los diferentes elementos del sistema HTCondor, desde los procesos de negocio hasta la infraestructura tecnológica que los soporta.

Esta vista consolidada facilita la comprensión del sistema en su totalidad y sirve como herramienta de comunicación entre los diferentes stakeholders del proyecto, permitiendo identificar puntos de integración críticos y oportunidades de optimización en toda la arquitectura.


\section{Diseño por capas de la solución}

\section{Capa de infraestructura}

\begin{figure}[H]
    \centering
    \includegraphics[width=\textwidth]{tablas-images/cp6/disenio-N1.png}
    \caption{Capa de Infraestructura}
\end{figure}

\section{Capa de virtualización}

\begin{figure}[H]
    \centering
    \includegraphics[width=\textwidth]{tablas-images/cp6/disenio-N2.png}
    \caption{Capa de Virtualización}
\end{figure}

\section{Capa de aplicación}
