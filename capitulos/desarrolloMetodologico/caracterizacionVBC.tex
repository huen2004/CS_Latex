\section{Nichos de mercado}

\subsection{Docker}
Docker se posiciona principalmente en el nicho de mercado de desarrolladores de software, empresas tecnológicas y proveedores de servicios en la nube que buscan una solución para la creación, implementación y gestión de aplicaciones en contenedores \citep{Hill2025}. Su capacidad de automatizar despliegues y garantizar la portabilidad entre entornos lo convierte en una opción ideal para DevOps y desarrollo ágil \citep{Mag2025}.

\subsection{Podman}
Podman está orientado a entornos empresariales y desarrolladores que requieren una solución de contenerización sin \textit{daemon}, compatible con OCI y con enfoque en la seguridad \citep{Surendhar2024}. Su naturaleza sin \textit{daemon} y su capacidad para ejecutar contenedores de forma aislada permiten su adopción en entornos donde la seguridad y la conformidad son prioridades \citep{Trevor2022}.

\subsection{Udocker}
Udocker se especializa en nichos de mercado académicos y de investigación, donde los usuarios necesitan ejecutar contenedores sin privilegios en sistemas que no permiten la instalación de software de nivel de sistema \citep{Campos2017}. Su facilidad para funcionar en entornos HPC (Computación de Alto Rendimiento) sin requerir permisos de root lo hace adecuado para instituciones de investigación \citep{Gomes2018}.

\subsection{Wasm (WebAssembly)}
Wasm se centra en el nicho de desarrollo web y aplicaciones de alto rendimiento en el navegador \citep{Haas2017}. Su capacidad para ejecutar código de forma eficiente en múltiples plataformas, incluidas aplicaciones de escritorio y móviles, lo convierte en una opción atractiva para empresas de desarrollo de software que buscan optimización multiplataforma \citep{Jangda2019}.

\subsection{LXC (Linux Containers)}
LXC es popular en entornos de virtualización ligera y servidores, donde se requiere un control granular sobre los entornos de contenedores \citep{Silva2024}. Su uso está orientado a proveedores de alojamiento web, desarrolladores de software y administradores de sistemas que necesitan un control preciso del entorno del sistema operativo \citep{Simon2023}.

\subsection{Containerd}
Containerd está dirigido a proveedores de servicios en la nube y plataformas de orquestación como Kubernetes, donde se requiere una solución de gestión de contenedores ligera y compatible con OCI \citep{Vano2023}. Su arquitectura modular lo convierte en una opción preferida para grandes infraestructuras \citep{Zhou2021}.

\subsection{LXD}
LXD se enfoca en nichos de mercado que requieren entornos de virtualización basados en contenedores que imiten máquinas virtuales, como proveedores de servicios en la nube, plataformas de pruebas y entornos de desarrollo \citep{Silva2024}. Su capacidad para ofrecer entornos de sistema completo lo hace ideal para desarrolladores y administradores de sistemas \citep{Kaiser2022}.

\subsection{Rkt}
Rkt fue diseñado para satisfacer las necesidades de proveedores de servicios en la nube y organizaciones que buscan una alternativa a Docker con un enfoque en la seguridad y compatibilidad OCI \citep{Lingayat2018}. Aunque su desarrollo ha sido discontinuado, sigue siendo relevante en entornos donde la compatibilidad y la seguridad son críticas \citep{Watada2019}.

\subsection{Singularity}
Singularity se centra en entornos de computación científica y HPC, donde se requiere portabilidad de aplicaciones sin necesidad de privilegios de root \citep{10.1145/3332186.3332192}. Es ampliamente adoptado en universidades, centros de investigación y laboratorios que ejecutan aplicaciones de alto rendimiento \citep{Kurtzer2017}.

\subsection{runC}
runC está orientado a proveedores de servicios en la nube, plataformas de orquestación como Kubernetes y desarrolladores de software que buscan una solución de contenedorización ligera y compatible con OCI \citep{Perez2005}. Su adopción en proyectos de gran escala se debe a su eficiencia y cumplimiento de estándares de contenedores \citep{151962df5f7e4b9faba0629540c11439}.

\subsection{CRI-O}
CRI-O está diseñado específicamente para su integración con Kubernetes, sirviendo como un motor de contenedores ligero y compatible con OCI para esta plataforma \citep{CNCF2019}. Es una solución ideal para proveedores de servicios en la nube y organizaciones que utilizan Kubernetes como su plataforma de orquestación principal \citep{151962df5f7e4b9faba0629540c11439}.

\subsection{Hyper-V Containers}
Hyper-V Containers están orientados a empresas que utilizan infraestructuras basadas en Windows, ofreciendo una solución de contenedorización segura y eficiente para aplicaciones basadas en Windows \citep{Smith2016}. Su integración con el ecosistema de Microsoft lo hace ideal para empresas con infraestructuras híbridas \citep{Clark2024}.

\subsection{OpenVZ}
OpenVZ se centra en proveedores de alojamiento web y servicios VPS, donde se requiere una solución de virtualización ligera basada en contenedores que permita un control granular sobre los recursos del sistema y la administración de múltiples instancias \citep{OpenVZ2015}.

\subsection{Linux VServer}
Linux VServer está orientado a administradores de sistemas y proveedores de servicios que requieren una solución de virtualización ligera basada en contenedores para la administración de servidores seguros y eficientes \citep{10.1145/1272996.1273025}. Es una opción adecuada para entornos de servidor dedicados y alojamientos compartidos \citep{LinuxVirt2017}.

\subsection{Google gVisor}
Google gVisor está dirigido a proveedores de servicios en la nube y organizaciones que priorizan la seguridad en sus entornos de contenedores \citep{LopezFalcon2024}. Su arquitectura de \textit{sandbox} proporciona un aislamiento fuerte, lo que lo convierte en una opción atractiva para aplicaciones sensibles \citep{gvisor2025}.

\subsection{Kata Containers}
Kata Containers se centra en entornos donde se requiere un alto nivel de seguridad y aislamiento, como proveedores de servicios en la nube y empresas que manejan información confidencial \citep{Viktorsson2020}. Su capacidad para combinar la eficiencia de los contenedores con el aislamiento de máquinas virtuales es su principal ventaja \citep{10.1145/1272996.1273025}.

\subsection{Firecracker}
Firecracker está orientado a proveedores de servicios en la nube y plataformas de cómputo en la nube que requieren micro VMs eficientes y seguras \citep{Jain}. Es una solución ideal para plataformas \textit{serverless} y entornos multi-tenant \citep{246288}.

\subsection{Sarus}
Sarus está dirigido a entornos de HPC y computación científica, donde los usuarios necesitan ejecutar contenedores de forma segura en sistemas de alto rendimiento \citep{Sarus2021}. Su compatibilidad con estándares de contenedores y su enfoque en la seguridad lo hacen ideal para centros de investigación y universidades \citep{B2020}.


\input{tablas-images/cp3/licencias.tex}

En la tabla~\ref{tab:interfaz-vbc} se puede ver la interfaz de uso por cada tecnología. Como se puede apreciar, la gran mayoría de tecnologías se utilizan a través de una CLI (línea de comandos). Esto, en muchos casos, puede implicar un aumento en la curva de aprendizaje, pero mayor facilidad para gestionar y automatizar las tecnologías una vez que se ha comprendido el uso de su interfaz.
\input{tablas-images/cp3/interfaz-uso.tex}

En el cuadro~\ref{tab:integracion-cloud-vbc} se puede ver la integración a las distintas plataformas cloud que tiene cada tecnología. Se puede ver que muchas tecnologías tienen integración directa con 3 de los proveedores cloud más famosos, a saber: AWS, GCP y Azure. Algunas tecnologías, por otro lado, solo soportan implementación de nubes privadas, como LXD.
\input{tablas-images/cp3/integracion-cloud.tex}

\section{Cuadrante Gartner}
Para visualizar el panorama de las tecnologias de contenerización en cuanto a su relevancia en la industria, se aplicó un cuadrante gartner, el cual mide dos dimensiones, Visión y Ejecución, es decir que tan buena proyeccion puede tener la tecnologia en el futuro y que tan bien se desempeña actualmente en el corto plazo.

\input{tablas-images/cp3/medicion-gartner.tex}

Una vez hecha la tabla anterior, se pueden resumir los resultados en el diagrama~\ref{fig:tabla-cuadrante-gartner}, que representa el cuadrante Gartner. Se pueden ver 4 grandes grupos de tecnologías dependiendo de su nivel de visión y ejecución: Líderes, Retadores, Visionarios, Jugadores de nicho.
\input{tablas-images/cp3/cuadrante-gartner.tex}

En la tabla~\ref{tab:entornos-ejecucion-vbc} se puede ver los sistemas operativos en los que se ejecutan estas tecnologias, se puede ver que la mayoria de estas estan diseñadas para sistemas linux, seguido de sistemas windows y por ultimo sistemas macOS, en cuyo caso las 2 unicas tecnologias que soportan este SO son ~\textit{Docker} y ~\textit{Podman}.
\input{tablas-images/cp3/entorno-ejecucion.tex}

En el cuadro~\ref{tab:matriz-dofa} se define una matriz DOFA, especificando las debilidades, oportunidades, fortalezas y amenazas de las tecnologías VBC. Esto permite tener varios aspectos en cuenta a la hora de seleccionar una tecnología o realizar una implementación.
\input{tablas-images/cp3/matriz-dofa.tex}
\input{tablas-images/cp3/documentacion.tex}