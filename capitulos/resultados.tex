\ChapterImageStar[cap:resultados]{Resultados}{./images/fondo.png}\label{cap:resultados}
\mbox{}\\
El desarrollo de esta investigación permitió obtener una serie de resultados tangibles y analíticos que responden a los objetivos planteados y contribuyen a la especificación de una solución arquitectónica basada en tecnologías de virtualización por contenedores (VBC) para el Grupo de Investigación en Redes, Información y Distribución (GRID) de la Universidad del Quindío. A continuación, se presentan los hallazgos más relevantes organizados según las fases metodológicas ejecutadas. En primer lugar, la caracterización del GRID permitió identificar sus capacidades tecnológicas actuales, su misión institucional y las necesidades específicas de sus stakeholders. Se determinó que el grupo cuenta con una infraestructura de virtualización basada en el hipervisor XCP-ng, compuesta por servidores tipo torre y rack, con especificaciones técnicas heterogéneas pero suficientes para soportar servicios educativos y de investigación. No obstante, se evidenció la necesidad de incorporar instancias computacionales más ligeras y eficientes que complementen la infraestructura existente y amplíen la oferta de servicios hacia la comunidad académica, en particular para los estudiantes de Ingeniería de Sistemas y Computación. El análisis de stakeholders permitió priorizar a los integrantes del GRID, docentes y estudiantes como actores clave, lo cual orientó el diseño de la solución hacia sus necesidades reales. La revisión sistemática de la literatura (SMS) arrojó como resultado la identificación y clasificación de 18 tecnologías VBC relevantes en el ámbito académico y profesional. Mediante una estrategia de búsqueda en bases de datos académicas y la técnica de bola de nieve, se seleccionaron y analizaron 116 estudios que permitieron construir un panorama actualizado del estado del arte. Tecnologías como Docker, Podman, LXC, Containerd, LXD, Singularity, Wasm, entre otras, fueron caracterizadas en términos de licencias, interfaces de uso, integración con nubes públicas, entornos de ejecución y adopción en la industria. La aplicación de un cuadrante Gartner permitió visualizar la posición relativa de cada tecnología, situando a Docker y Containerd en el cuadrante de líderes, mientras que otras como Udocker o Sarus se ubicaron como jugadores de nicho, orientados a entornos específicos como HPC. El benchmarking técnico realizado sobre un subconjunto de tecnologías preseleccionadas —Docker, Podman, LXC, LXD y Containerd— proporcionó mediciones objetivas de rendimiento en condiciones controladas. Los resultados mostraron diferencias significativas en el consumo de recursos, tiempo de arranque, throughput de red y latencia de E/S. Containerd destacó por su eficiencia general, con un bajo consumo de CPU (99.71\% de tiempo en idle), un tiempo de arranque rápido y una latencia de disco reducida. LXC, por su parte, mostró el menor consumo de memoria RAM (2.835\%) y el mejor desempeño en throughput de red, aunque con un tiempo de arranque superior. Podman, si bien eficiente en recursos, presentó limitaciones en el rendimiento de red y E/S. Estos resultados proporcionaron una base cuantitativa para la toma de decisiones subsiguiente. El Análisis de Decisiones y Resolución (DAR), aplicado conforme al modelo CMMI, permitió evaluar las tecnologías VBC con base en 12 criterios técnicos y organizacionales, entre los que se incluyeron: tipo de licencia, posibilidad de orquestación, compatibilidad con Docker Hub, soporte para redes personalizadas, persistencia de datos, documentación, soporte comunitario, popularidad, consumo de recursos, compatibilidad con orquestadores y costos de implementación. Containerd obtuvo la puntuación más alta (233/300), destacando por su licencia Apache 2.0, integración nativa con Kubernetes, amplia documentación, bajo consumo de recursos y compatibilidad total con el ecosistema Docker. Adicionalmente, se evaluaron distintas alternativas de motores de orquestación para Kubernetes, donde K3S resultó seleccionado por su facilidad de instalación, bajo consumo de recursos y compatibilidad con entornos limitados en hardware. Con base en lo anterior, se diseñó una solución arquitectónica modelada en Archimate, que integra la infraestructura existente del GRID con una capa de virtualización basada en Containerd y K3S. El modelo incluye vistas de negocio, aplicación y tecnología, articulando procesos como la solicitud de entornos virtualizados, la provisión de recursos y la gestión de ejecuciones. Se propuso un modelo por capas que separa claramente la infraestructura física, la capa de virtualización (contenedores y orquestación) y la capa de aplicación, facilitando así el mantenimiento, la escalabilidad y la gestión de servicios. Finalmente, se implementó y validó un producto mínimo viable (PMV) que materializó la arquitectura propuesta. El PMV fue desplegado en un entorno controlado dentro de la infraestructura del GRID, permitiendo la ejecución de contenedores OCI gestionados mediante K3S y Containerd. La validación incluyó pruebas de funcionalidad, rendimiento y usabilidad, realizadas en colaboración con usuarios potenciales (investigadores y estudiantes). Los resultados de estas pruebas confirmaron que la solución es estable, reproducible y adecuada para su uso en escenarios académicos y de investigación, cumpliendo con los requerimientos identificados durante la caracterización inicial. En conjunto, estos resultados demuestran que la incorporación de Containerd y K3S en la infraestructura del GRID representa una alternativa viable y eficiente para expandir su portafolio de servicios mediante contenedores, sin desestimar la inversión previa en virtualización tradicional. La metodología aplicada —que combinó caracterización contextual, revisión sistemática, evaluación técnica y diseño arquitectónico— provee un marco reproducible para la selección e implementación de tecnologías VBC en contextos académicos con restricciones de recursos y alineados con misiones institucionales de docencia, investigación y extensión.