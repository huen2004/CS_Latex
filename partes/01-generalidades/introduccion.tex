\ChapterImagePrelim[cap:introduccion]{Introducción}{./images/fondo.png}\label{cap:introduccion}
\mbox{}\\

La computación científica se enfoca en la solución de problemas complejos que, por su naturaleza o escala, desbordan la capacidad de resolución analítica humana~\citep{landau01}. En este campo, el ordenador no es solo una herramienta, sino un recurso indispensable que permite modelar y analizar fenómenos del mundo real que de otro modo serían inaccesibles. La capacidad de procesamiento de las computadoras modernas abre la puerta a la exploración de problemas que, aunque teóricamente solucionables, en la práctica demandan un volumen de cálculo extraordinario \citep{landau01}.

Sin embargo, la potencia de un único equipo es a menudo insuficiente. Ciertos desafíos científicos, caracterizados por conjuntos de datos masivos o una complejidad inherente, hacen inviable su ejecución en una sola máquina. Para superar esta barrera, la comunidad científica recurre a la computación de alto rendimiento (HTC, por sus siglas en inglés: \textit{High-Throughput Computing}), un paradigma de la computación distribuida cuyo objetivo es maximizar el número de tareas completadas en un período determinado \citep{Juve2015}. Este enfoque no solo es fundamental para la investigación, sino que también ha despertado un creciente interés en el ámbito educativo como herramienta pedagógica \citep{Senol-01}.

En este ecosistema tecnológico destaca HTCondor, un sistema de gestión de cargas de trabajo desarrollado por la Universidad de Wisconsin–Madison y optimizado para el cómputo intensivo \citep{chang-01, htcondor-description}. HTCondor facilita el envío de trabajos a un clúster de computadoras, administrando de manera autónoma la asignación de recursos, la planificación y la distribución de las tareas entre los nodos disponibles. Su innovador mecanismo de planificación se basa en un modelo de políticas bidireccional, donde tanto los proveedores de recursos como los usuarios pueden definir preferencias y restricciones que rigen la ejecución de los trabajos \citep{htcondor-description}.

Las tareas computacionales en HTCondor se organizan en ``universos'', que no son más que entornos de ejecución predefinidos para distintos tipos de lenguajes o tecnologías. En el momento de la redacción de este documento, los universos soportados por HTCondor incluyen:~\textit{vanilla, grid, java, scheduler, local, parallel, vm, container y docker.}
